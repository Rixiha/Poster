\RequirePackage{xcolor}
\documentclass{sciposter}
\usepackage{multicol}
\usepackage{graphicx,url}
\usepackage[spanish]{babel}   
\usepackage[utf8]{inputenc}

\title{Simulación de epidemias bajo medidas de contingencia}
\author{Avril Paola Mejía Avianeda y Elisa Schaeffer}
\institute {Posgrado en Ingeniería de Sistemas}
\email{amejiaa1900@alumno.ipn.mx}

\leftlogo[1]{uanl.png} 
\rightlogo[1]{fime.png}

\begin{document}

\conference{\raisebox{2cm}[0cm]{\includegraphics[width=50mm]{qr-code.png}}
  \hfill
  \raisebox{2cm}[0cm]{\includegraphics[width=50mm]{ipn.png}}
  \hfill
  \raisebox{2cm}[0cm]{\includegraphics[width=150mm]{verano2021.png}}}

\maketitle

\section{Introducción}
La humanidad ha sido asediada por enfermedades infecciosas a lo largo de la historia. Ejemplos en la era moderna incluyen las epidemias del SARS, MERS, influenza AH1N1, ébola, y en la actualidad, el SARS CoV-2, virus que causa la enfermedad conocida como covid-19. Ante estas eventualidades, gobiernos de distintos niveles deben adoptar medidas prontas y efectivas para evitar una crisis de salud pública. Sin embargo, es difícil saber el impacto que tendrán las acciones tomadas ante un sistema complejo y dinámico, como lo es la propagación de una enfermedad en la población. Ante la inviabilidad logística, y quizá ética, de ensayar distintas medidas diréctamente a nivel población, surge la necesidad de realizar ensayos computacionales mediante modelos matemáticos de la enfermedad. La naturaleza aleatoria y evolutiva de los procesos de contagio hace de las simulaciones estocásticas una de las maneras más efectivas de estudiar y predecir el fenómeno.

Las técnicas de simulación multi-agente permiten analizar y cuantificar los efectos de distintas medidas ante la propagación de enfermedades, tales como el distanciamiento social, el uso de cubrebocas, o el aislamiento social, además de interacciones con otros factores como la densidad poblacional, nivel socioeconómico y la calidad de aire. La comprensión de estas diferencias conlleva a una toma de decisiones facilitada y basada en evidencia científica. Esta propuesta continúa la investigación iniciada en los proyectos PAICyT IT512-15 \emph{Herramientas computacionales para análisis epidemiológico multifactorial} y PAICyT CE1421-20 \emph{Exploración algorítmica de relaciones entre calidad de aire y bienestar}.

Como hipótesis se tiene que la simulación de modelos epidemiológicos permite una toma de decisiones más informada y con mejores resultados.\par

El objetivo general es diseñar, implementar y analizar una simulación multi-agente epidemiológica que permita medir los efectos que tienen distintas medidas de contención contra el contagio y propagación de una enfermedad infecciosa.\par
Los \emph{objetivos específicos} para el presente proyecto son:\par
\begin{itemize}
  \item \textbf{Modelado matemático} Modelado matemático. Diseñar e implementar una simulación multiagente de un modelo epidemiológico.
  \item \textbf{Software abierto} Implementar un prototipo computacional para explorar los efectos de distintas medidas de contingencia.
  \item \textbf{Visualización científica} Cuantificar y visualizar los efectos de las diversas medidas para evitar la propagación del virus.
\end{itemize}

Motivación.

justificación. 

\section{Antecedentes}
Los modelos matemáticos para el estudio de epidemias han sido estudiados por décadas (Bailey, 1975; Britton, 2010). Varios buscan predecir el tamaño final de una epidemia con alguna probabilidad, así como otros buscando controlar el contagio (Nowzari, Preciado, y Pappas, 2016), mientras otros han estudiado el impacto de las medidas de contención de la
propagación del virus (Fransson y Trapman, 2019). En el caso específico de las simulaciones multi-agente, además de ser usadas para el estudio de epidemias (Hassin, 2021; Hoertel y cols., 2020a; Perez y Dragicevic, 2009; Venkatramanan y cols., 2018) también se han usado para abordar problemas de transporte (Hörl, 2017) o finanzas (Samitas, Polyzos, y Siriopoulos, 2018). Nuestro gobierno no está exento de los retos que presenta enfrentar una crisis sanitaria de naturaleza epidémica, y tomar la decisión equivocada puede tener costos exorbitantes tanto en materia económica como en vidas humanas (Lipsitch, Finelli, Heffernan, Leung, y Redd, 2011; Maringe y cols., 2020; Pasquini-Descomps, Brender, y Maradan, 2017). Proyectos como el que proponemos pueden ayudar a que se entiendan mejor los impactos de las medidas tomadas, ya sea para contener sus efectos o para tomar una decisión más informada.


\section{Estado de arte}

Qué han hecho los demás sobre este tema (citar a publicaciones científicas, de preferencia publicadas en revistas que tengan DOI y que por lo menos algunos sean de los últimos cinco años.

Área de oportunidad: qué exactamente este trabajo contribuirá encima de lo que ya existe.  {\textquestiondown}Qué tiene de diferente/original/impacto?

\section{Solución propuesta}
Se busca implementar modelos del comportamiento de epidemias en una población mediante técnicas de simulación multi-agente, incorporando medidas de contingencia. El objetivo es identificar hasta qué grado diversos factores propician o disminuyen el número de contagios, y con ello, apoyar la toma de decisiones de salud pública.

Como herramienta principal se tiene NetworkX, el cual forma parte de la paquetería de Phyton para la creación, manipulación y estudio de la estructura, dinámicas y funciones de redes complejas.

\section{Experimentos}

\section{Conclusiones}

\paragraph{Agradecimientos}

{\small Delfín. Agradecer a las demás personas que \underline{no son
    autores} quienes ayudaron en algo. El póster se preparó con \url{https://www.overleaf.com/}.}

\end{document}
